\chapter{Power structure}
\section{Rails description}
An overview of the power structure for \theproject is shown in \cref{fig-power-overall}.
The main power rails specifications are :
%\begin{table}[htbp]
%    \centering
%    \begin{tabular}{|c|l|c|c|c|c|p{4cm}|}
%        \hline
%        \headerrow Rail name & Characteristic & Min. & Typ. & Max & Unit & Remarks \\\hline
%        VCAP & Voltage & $\approx 5$ & 14 & 18 & V & Depends on the load \\\hline
%        VBAT & Voltage & 9 & 9.9 & 10.8 & V & Depends on the load \\\hline
%    \end{tabular}
%\end{table}
\begin{itemize}
    \item \textbf{VCAP}: Main rail. Powered by the solar cells through MPPT, with backup from the capacitor banks and/or the LiFePO4 batteries.
    The maximal voltage depends on the caps and SAB system, but can be safely assumed to be 18V. Under 5V, the car systems are deemed powered down.
    \item \textbf{VBAT}: Battery rail. Either powered from VCAP (while charging), powering VCAP when VCAP < VBAT (and potential command) or virtually disconnected.
    \item \textbf{+5V}: Standard logic rail. Power the radio transciever, the transponder and the direction servomotor. Does not require a high level of stability.
    \item \textbf{+3.3V}: Low voltage and custom logic rail. Power the MCU, the analog and all other logic.
    \item \textbf{+12V}: AUX power rail. 12V nominal, can actually be anywhere between +5V and +12V. Used for gate drive and buzzer drive. No strict requirement on stability.
\end{itemize}
\begin{landscape}
    \begin{figure}
        \centering
        \def\svgwidth{\linewidth}
        \input{pdftex/overall_power_arch.pdf_tex}
        \caption{Overview of the power supply scheme}
        \label{fig-power-overall}
    \end{figure}
\end{landscape}
