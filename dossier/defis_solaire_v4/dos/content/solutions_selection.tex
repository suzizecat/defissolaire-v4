\chapter{Technical solutions}
\section{Power supplies}
\subsection{Input MPPT Buck/Boost}
Use of dedicated chip, with external inductor and MOSFETs.
The design should consider up to 6A and 8.12V at the input and should output up to the maximum value of the supercaps stack, minus a safety margin.
Moreover, the DC/DC should be able to work both in CC and CV.

Finally the converter shall handle high impedance power source, thus providing MPPT.
A seen feature is to reduce the output power when the input voltage drops (e.g. \href{https://www.analog.com/en/products/lt8710.html}{Analog Devices LT8710}).
Other vendors also have supercaps chargers with MPPT feature.

The buck feature can be avoided by bypassing the controller when the input voltage is over the output voltage and avoiding consumption if VCELL > VCAP.

\subsection{5V Buck converter}
In order to avoid the loss of a big voltage gap between the VCAP and the 5V rail, a buck converter would probably be best.
As of today, this feature is managed by a TracoPower TSRN2450 which handle the task reliably.

\subsection{3.3V LDO converter}
To provide the 3.3V, a LDO is chosen in order to get a rail as clean as possible.
The 5V to 3.3V drop is manageable and the power is low, as the current should be under 50mA at worst.

Therefore, the loss is between 85mW (for a 50mA) or 130mW for a worst case 100mA consumption.

\paragraph{}
It might be interesting to evaluate a SMPS, but there might be a risk of a noisy 3.3V rail.

\subsection{12V Rail}
The target is to get the behavior shown in \cref{fig-12v-behav}
The rail should roughly follow VCAP, but clamped at 12V.
\begin{figure}[htbp] 
    \centering
    \input{pdftex/12v_rail_behav.pdf_tex}
    \caption{Expected ideal 12V rail voltage depending on VCAP voltage}
    \label{fig-12v-behav}
\end{figure}

As the peripheral will either have some milliamps continuously for few seconds max (buzzer)
or short pulses of high current (100ns max at 2A max (example)), the average power should be low.

Hence the choice could be a standard linear regulator, potentially with an accent on the low quiescent current aspect.

\paragraph{}
Otherwise, a buck SMPS, with low output power feature and some kind of management of the situation where Vin < Vout could be considered.

The voltage on this rail may not be very stable without particular issue as long as it doesn't fall bellow the Vth of mosfets.

\subsection{Battery management}
The target would be to have 3 or 4 LiFePO4 or LiPo cells, used when the solar cell cannot supply the required power.
The charge time will be 30 min under maximal sunlight plus, maybe some time during the race.
Using 4S LiFePO4 for reference (11.12 V), we can safely assume 3A continuous at the output of the MPPT, giving 1.5Ah.
Considering "standard" capacity values, a 2.2Ah pack would fit those requirements fairly well. 

Given that the safe charging rate is 1C, potentially going up to 2C, the maximum charging current shall be lower or equal to 4.4A

The battery management system shall provide charging and balancing capacity (potentially using two different components) as well as a mean to prevent over-discharge.

\paragraph{}
The discharge current can be much higher (sometime announced at up to 30C). 
But given that the motors have a combined coil resistance of 3 Ohms (6 ohms by two motors), this gives a 2C discharge rate which seems safe.
The system shall therefore be able to handle 4A from battery to VCAP.

\paragraph{}
The idea is to use power from the battery as little as possible and only when there is no sunlight or there is a transient need for power backup (20\% slope)
To avoid over discharge, it is required to, at least, cut-off the path from VBAT to VCAP. 